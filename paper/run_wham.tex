% !TEX root = ./paper.tex

The most common way to extract a free energy profile from umbrella sampling data
is the weighted histogram analysis method, or WHAM \cite{Swendsen-1995, Roux-1995, Swendsen-1992}.
For the most common case, where all
trajectories were run at the same temperature and the bias is solely a function
of the collective variable, the two key WHAM equations are

EQUATIONS HERE, USING MY OLD NOTATION

where DEFINE TERMS HERE. Each trajectory is histogrammed along the chosen
collective variable, and thus produced a biased estimate of the probability
distribution. If there were only one trajectory, the WHAM equations would reduce
to simple thermodynamic reweighting, but with multiple estimates from the
various trajectories one must compute some kind of weighted average.  The  $F_i$
values serve precisely this purpose; the WHAM derivation shows that iteratively
solving these equations minimizes the variance in the resulting probability
distribution.

There are a number of WHAM implementations available.  Probably the most commonly used one is due to Grossfield \cite{WHAM}, but there are also version included as part of gromacs \cite{Lindahl-2013} and CHARMM \cite{CHARMM2009}.

Although WHAM is a robust method, there are a number of checks one should apply
to any calculation performed with it (this is in addition to the design-level
tests discussed in Section \ref{s:precheck}).

\begin{itemize}

    \item \textbf{Check that your units are consistent}.  Many WHAM codes are
    agnostic to the choice of collective variable, but do require that the units
    be consistent.  Specifically, most implementations assume the restraints are
    harmonic in the reaction coordinate, and thus expect that if the time series
    are given in units of $X$ (e.g. {\AA} or nm for a distance), the restraint's
    spring constant is given in units of $\mathrm{energy}/X^2$, and will output
    the free energy curve in units of $\mathrm{energy}$.  There are known issues
    with some simulation packages. Most notably, when restraining angles or
    torsions in Amber, the location of the restraint minimum is specified in
    degrees (as is the output), but the restraint is specified in radians.  This
    mismatch is dramatic enough to cause the WHAM calculation to fail
    immediately, generally producing all ``NaN'' values for the free energy
    curve.

    \item \textbf{Check the functional form of the restraint}.  Although most
    packages use harmonic restraints, some include a prefactor of 0.5, while
    others do not.  To produce correct results, you may need to given a
    different value to the WHAM code than you did to the simulation package.
    For example, NAMD and Amber restraints restraints do not use the prefactor,
    while Grossfield's WHAM does, so if you ran a NAMD simulation with a spring
    constant of 10 kcal/{\AA}$^2$, you would put 20 as the spring constant in
    the WHAM metadata file.

    \item \textbf{Verify the histograms overlap}.  If there are ``bare'' patches
    (or even just poorly sampled regions), the free energy profile will not be
    reliable.  This test should be performed \emph{before} running WHAM; WHAM
    works best when the total number of samples is roughly constant across the
    whole range of the collective variable.

    \item \textbf{Make sure the iteration is fully converged}.  The WHAM
    equations are generally solved by iteration to self-consistency, so it is
    crucial to ensure that the iteration has proceeded far enough that the
    answer is not changing.  You should check the manual for your chosen WHAM
    implementation for a suggested value, but also should (if possible) plot the
    intermediate free energy curves to verify they've stopped changing.  If the
    range of values on your free energy curve is especially large (hundreds of
    $k_B T$), there may be significant issues with numerical precision
    preventing the curves from converging.  In this case, you can use a
    special-purpose WHAM implementation intended to handle this case, available
    from \url{https://github.com/dejunlin/gwham}.

    \item \textbf{Check that the chosen bin width isn't altering the answer}.
    Finite-width histogram bins intrinsically smooth the free energy profile by
    breaking it into regions.  This effect can be minimized (and often
    effectively eliminated) by reducing the bin size, at the cost of reducing
    the number of samples in each bin, which makes the curve noisier.  To verify
    that the bin width isn't a problem, one should always re-run WHAM several
    times with different bin width to verify that the resulting curve does not
    vary systematically.  The key is that the bins must be narrower than the
    relevant ``features'' of the free energy curve; bins comparable to or
    broader than these features with obscure them or even smooth them away.

\end{itemize}

It is also crucial to assess whether other relevant orthogonal degrees of
freedom (variables other than the one used as a collective variable) are
well-sampled.  This is a complex general problem, so we can't give you a simple
test.  Rather, you have to honestly look at your system, and verify that
overlapping trajectories are genuinely similar (and not just similar according
to the collective variable).  The only automated procedure for looking for
problems that we are aware of is due to Neale et al \cite{Pomes-2014}, and only
applies to the case where the umbrella sampling was performed using Hamiltonian
replica exchange; in essence, the method looks for regions of the collective
variable where there is hysteresis in the migration of the trajectories up and
down the variable range as evidence for slow orthogonal relaxation.
