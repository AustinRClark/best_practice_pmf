% !TEX root = ./best_pmf.tex

\label{s:precheck}

% Note from AMG: I'm putting a lot of text in the checklist, but
% we might want to break that text out into the body of the document.

The first key to a successful outcome is appropriate experimental design.  That is, you first need to ensure that the calculation is plausibly doable given the resources available, and that the question is is well-posed.  The following is a list of items that must be considered before beginning such a calculation.  Answering these questions will help you decide which techniques

\begin{itemize}

    \item \textbf{What collective variable or variables will be used as reaction
    coordinate?}  The statistical physics of free energy curves gives us
    considerable leeway in choosing our reaction coordinates --- in principle,
    any variable could be used, and if the calculation is performed correctly
    the resulting free energy curve will be ``true''.  However, as a practical
    matter, interpreting the curve will be challenging (or even deceptive)
    unless the reaction coordinate is a good approximation to the true
    mechanism.

    \item \textbf{What are the other relevant motions in the system? On what
    timescale do they take place?}  The derivation of a free energy curve
    involves computing a thermodynamic average over all degrees of freedom in
    the system other than the chosen path.  In practical terms, this means that
    any slow degrees of freedom in your system that aren't explicitly biased or
    tracked must be sampled for the results to converge statistically.  This is
    particularly problematic if there are multiple slowly exchanging states not
    explicitly spanned by the chosen collective variable.

    \item \textbf{What is the expected ``lengthscale'' of features on the
    reaction coordinate?}  Over what range will the collective variables be
    tracked? How finely do we need to determine the free energy curve to be able
    to answer the scientific question?

    \item \textbf{Can the reaction coordinate be used to calculate a biasing
    energy or force? }  Many techniques, such as metadynamics and umbrella
    sampling, make direct use the chosen collective variables by adding
    additional biasing forces to the simulation.  Thus, when using one of these
    techniques, one is restricted to collective variables that can easily be
    computed on the fly in the simulation, preferably without greatly reducing
    computational performance.

    \item \textbf{How many collective variables do you plan to bias?} There's a
    challenging tradeoff here: If there are slow degrees of freedom you don't
    bias, you may effectively just be doing brute force sampling.  On the other
    hand, the number of trajectories typically increases exponentially with the
    number of collective variables biased.  As a result, it is very rare that you see more than 1 or 2 dimensional enhanced sampling.

    \item \textbf{Are the barriers expected to enthalpic, entropic, or both? Are
    there major entropic differences between states?}  Different enhanced
    sampling methods do better with different kinds of biases.  For example,
    conventional replica exchange is great for purely enthalpic barriers
    (raising the temperature exponentially increases the rate of barrier
    crossing) but is less helpful with entropic ones (the only gain at high
    temperature is an increase in the ``diffusion'' constant).  Thinking about
    the kinds of barriers involved should inform your choice of enhanced
    sampling technique.

\end{itemize}
