% !TEX root = ./paper.tex

\section{Checklist for designing a free energy profile calculation}

% Note from AMG: I'm putting a lot of text in the checklist, but
% we might want to break that text out into the body of the document.

The first key to a successful outcome is appropriate experimental design.  That is, you first need to ensure that the calculation is plausibly doable given the resources available, and that the question is is well-posed.  The following is a list of items that must be considered before beginning such a calculation.  Answering these questions will help you decide which techniques

\begin{itemize}

    \item \textbf{What collective variable or variables will be used as reaction
    coordinate?}  The statistical physics of free energy curves gives us
    considerable leeway in choosing our reaction coordinates --- in principle,
    any variable could be used, and if the calculation is performed correctly
    the resulting free energy curve will be ``true''.  However, as a practical matter, interpreting the curve will be challenging (or even deceptive) unless the reaction coordinate is a good approximation to the true mechanism.

    \item \textbf{What are the other relevant motions in the system? On what timescale do they take place?}

    \item \textbf{What is the expected ``lengthscale'' of features on the reaction coordinate?}

    \item \textbf{Can the reaction coordinate be used to calculate a biasing energy or force? }

    \item \textbf{Are the barriers expected to enthalpic, entropic, or both? Are there major entropic differences between states?}  I'm thinking here of folding or oligomerization, where there's a big entropic barrier to finding one of the states.

\end{itemize}}
